%!TEX TS-program = xelatex
%!TEX encoding = UTF-8 Unicode
% Awesome CV LaTeX Template
%
% This template has been downloaded from:
% https://github.com/posquit0/Awesome-CV
%
% Author:
% Claud D. Park <posquit0.bj@gmail.com>
% http://www.posquit0.com
%
% Template license:
% CC BY-SA 4.0 (https://creativecommons.org/licenses/by-sa/4.0/)
%


%%%%%%%%%%%%%%%%%%%%%%%%%%%%%%%%%%%%%%
%     Configuration
%%%%%%%%%%%%%%%%%%%%%%%%%%%%%%%%%%%%%%
%%% Themes: Awesome-CV
\documentclass[11pt, a4paper]{awesome-cv}

% Configure page margins with geometry
\geometry{left=1.4cm, top=.8cm, right=1.4cm, bottom=1.8cm, footskip=.5cm}

% Specify the location of the included fonts
\fontdir[fonts/]

%%% Configure a directory location for sections
%\newcommand*{\sectiondir}{resume/}

%%% Override color
% Awesome Colors: awesome-emerald, awesome-skyblue, awesome-red, awesome-pink, awesome-orange
%                 awesome-nephritis, awesome-concrete, awesome-darknight
%% Color for highlight
% Define your custom color if you don't like awesome colors
\colorlet{awesome}{awesome-darknight}
\definecolor{awesome}{HTML}{222222}
%% Colors for text
\definecolor{darktext}{HTML}{333333}
\definecolor{text}{HTML}{222222}
\definecolor{graytext}{HTML}{0457A0}
\definecolor{lighttext}{HTML}{666666}

% Set false if you don't want to highlight section with awesome color
\setbool{acvSectionColorHighlight}{true}


% If you would like to change the social information separator from a pipe (|) to something else
\renewcommand{\acvHeaderSocialSep}{\quad\textbar\quad}

%%% Essentials

%\photo[circle,noedge,left]{./gaurav_agarwal.png}
\name{Gaurav Agarwal}{}
%\address{\#401, Mathura Villa, Tapadia complex, Station Road, Jamshedpur}
\mobile{(+91) 805-083-7120}
%%% Social
% \gitlab{gitlab-id}
%\stackoverflow{SO-id}{SO-name}
\twitter{@compnaut}
\email{compnaut@gmail.com}
%\homepage{www.aboutme.com}
\github{compnaut}
\linkedin{compnaut}

%\quote{``2+ years experience specializing in the embedded development life cycle, automation, and testing. Interested in devising a better problem-solving method for challenging tasks, growing with learning new technologies and tools if the need arises."}
%%%%%%%%%%%%%%%%%%%%%%%%%%%%%%%%%%%%%%
%     Content
%%%%%%%%%%%%%%%%%%%%%%%%%%%%%%%%%%%%%%

\begin{document}
%%% Make a footer for CV with three arguments(<left>, <center>, <right>)

\makecvheader[C]

\makecvfooter
  {\today}
  {Gaurav Agarwal~~~·~~~Résumé}
  {\thepage}

%%% Import contents
\cvsection{Education}
\begin{cventries}
	\cventry
	{B.E in \href{https://eee.pes.edu/}{Electrical and Electronics Engineering}}
	{\href{https://pesit.pes.edu/}{P.E.S Institute of Technology, Autonomous Institute under VTU, Belgaum}}
	{Bangalore, India}
	{Aug.'13 - May'17}
	{
		\begin{cvitems}
		\item{GPA: 8.95/10.00}
		\item{Major Courses: Embedded System, Control Systems, Digital Signal Processing}
		\end{cvitems}
	}
\end{cventries}
\begin{cventries}
	\cventry
	{I.C.S.E, I.S.C in {Pure Science with Computer Application}}
	{\href{https://ksms.ac.in/}{Kerala Samajam Model School}}
	{Jamshedpur, India}
	{Mar'99 - May'13}
	{
		\begin{cvitems}
		\item{ISCE: 93.4\%, ISC: 88.75\%}
		\end{cvitems}
	}
\end{cventries}

\cvsection{Work Experience}
\begin{cventries}
	\cventry
	{Team Indus Skywalker, Flight Software | Integrated Avionics | Command \& Data Handling}
	{\href{http://www.teamindus.in/}{Team Indus (Axiom Research Labs Pvt. Ltd.)}}
	{Bangalore, India}
	{Jul.'17 - Present, \textbf{Intern}: Jan.'17 - Jun.'17}
	{
		\begin{cvitems}
		\item{Developing software systems for \textbf{orbital}, \textbf{descent} and \textbf{surface} phases of the \textbf{Z01} lunar mission, with onboard state estimation, autonomous attitude correction, lunar terrain feature tracking, along with interface drivers for the \textbf{On-Board Computer (OBC)}\href{http://ww1.microchip.com/downloads/en/DeviceDoc/ATF697FF.pdf}, interfacing cards and sensors, and limited fault detection, isolation, and recovery.}
			\item{Design and testing of telecommand packet definition for the entire lunar landing mission: real time, absolute time tagged, patch, differential time tagged, configurable block and event based commands.}
			\item{Developing OBC boot architecture, custom linkers, make system and maintaining different configurations for Atmel's RAD Hard \textbf{ATF697FF} - \textbf{SPARC V8} processor operated as bare metal with \textbf{Round Robin} scheduler}
			\item{Developed framework for \textbf{Processor in Loop Simulation (PiLS)} system emulating sensor and actuator electrical interfaces to lander avionics unit. Responsible for regular comprehensive PiLS exercises of lunar lander avionics system.}
			\item{Developing frameworks for regression unit, interface and integration level of testing which involves sensor and other interface cards emulation and simulation using \textbf{Interface Emulation Card (IEC) }\href{http://zedboard.org/product/microzed/} and OBC.}
		\end{cvitems}
	}
\end{cventries}

\cvsection{Academic Projects}
\begin{cventries}
	\cventry
	{\textbf{PISAT}\href{http://pisat.pes.edu/}- a nano-satellite project executed by \textbf{CoRI}\href{http://cori.pes.edu/}, P. E. S. University launched aboard PSLV C-35 on 26th Sept'16}
	{Student Team Lead}
	{Bangalore, India}
	{Oct.'14 - Dec.'16}
	{
		\begin{cvitems}
		\item{Involved in complete design, development, assembly, integration and testing phase of \textbf{PISAT}\href{https://www.isro.gov.in/Spacecraft/pisat} - a nano-satellite under ex-ISRO experts.}
		\item{OBC and ADCS: The subsystem included building real time software for an imaging satellite in a component base manner which managed overall functionality such as attitude determination, control systems, telemetry and telecommand (RTE) on a Atmel \textbf{AT32UC3A0512 }\href{https://www.microchip.com/wwwproducts/en/AT32UC3A0512} micro-processor with bare metal architecture using Round Robin scheduler.}
		\item{Payload: Develop \textbf{NanoCam C1U }\href{https://gomspace.com/UserFiles/Subsystems/datasheet/gs-ds-nanocam-c1u-17.pdf} functionality, operations and test bench for complete analysis of setting of the camera parameters.}
		\item{Assembly, Integration and Testing: Build robust test system which was used for Avionics bring ups, \textbf{On-board in Loop Simulation (OiLS)}, independent verification of telemetry, telecommand, payload interface and ground checkout.}
		\item{Mission Planning and Operations: Reviewing and making of the detailed design documents for CDR, PSR, PLR, sequence of events, PISAT in orbit tracking and post data analysis.}
		\end{cvitems}
	}
\end{cventries}

\cvsection{Program Committees}
	\begin{cvskills}
	\cvskill
		{Member}
		{\textit{\textbf{Team Indus Foundation}}: Giving talks, organize technical workshops for school/colleage students}
	\cvskill
		{Core Member}
		{\textit{\textbf{IEEE Student Branch }}\href{http://ieee.pes.edu/}, P. E. S University: Organize various technical workshops, talks, and designed yearly magazines.}
	\cvskill
		{Core Member}
		{\textit{\textbf{Collegiate Social Responsibility Club (CSR)}}\href{http://pes.edu/clubs/pes-csr-club/}, P. E. S. University: Organize various social welfare activities.}
	\cvskill
		{Secretary}
		{\textit{\textbf{Space Research Club (SPARC)}}\href{https://www.facebook.com/Space-Research-Club-Of-PES-University-487153011445398/?tn-str=k*F}, P. E. S. University: To excite students for collaborative projects/competitions in space domain}
	\cvskill
		{Event Organizer}
		{\textit{\textbf{PES Annual Cultural- Techno Fest AatmaTrisha}}\href{http://pes.edu/clubs/atmatrisha/}: Organize various technical and informal events}
	\cvskill
		{Presenter | Organizer}
		{\textit{\textbf{Prakalpa PES Annual Exhibition}}\href{http://pes.edu/clubs/prakalpa/}: For CoRI and IEEE stalls}
	\cvskill
		{Founder and Mentor}
		{\textit{\textbf{RISE Academy}}, PES Boys Hostel: Initiative to impart skills among juniors and organize hostel fests}
\end{cvskills}

\cvsection{Awards and Accolades}
\begin{cvhonors}
	\cvhonor
	{\textbf{APCOSEC'16}- Asia Oceania Systems Engineering Conference}
	{Published a paper titled "Design of a student satellite -PISAT"}
	{Bangalore, India}
	{Jan'16}
	\cvhonor
	{Bronze Award in System Engineering Challenge organized by \textbf{INCOSE}}
	{Presented a Paper Titled "Telemetry and telecommand for PISAT"}
	{Bangalore, India}
	{May'16}
\end{cvhonors}


\end{document}
