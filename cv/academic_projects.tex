\cvsection{Academic Projects}

\begin{cventries}
	\cventry
	{\textbf{PISAT}\href{http://pisat.pes.edu/}- a nano-satellite project executed by \textbf{CoRI}\href{http://cori.pes.edu/}, P. E. S. University launched aboard PSLV C-35 on 26th Sept'16}
	{Student Team Lead}
	{Bangalore, India}
	{Oct. 2014 - Dec. 2016}
	{
		\begin{cvitems}
		\item{Involved in complete design, development, assembly, integration and testing phase of \textbf{PISAT}\href{https://www.isro.gov.in/Spacecraft/pisat} - a nano-satellite student project funded by ISRO and PES University. Worked in following subsytems under the expertize of ex-ISRO scientists:}
		\item{System Engineering: Subsystem level requirements collation, design and development life cycle, complete verification and validation for both hardware and software.}
		\item{OBC and ADCS: The subsystem included building real-time software for an imaging satellite in a component base manner which managed overall functionality such as attitude determination, control systems, telemetry and telecommand (RTE) on an Atmel \textbf{AT32UC3A0512 }\href{https://www.microchip.com/wwwproducts/en/AT32UC3A0512} micro-processor with bare-metal architecture. Build test frameworks for scenario-based testing, open-loop and closed-loop simulations.}
		\item{Payload: Develop \textbf{NanoCam C1U }\href{https://gomspace.com/UserFiles/Subsystems/datasheet/gs-ds-nanocam-c1u-17.pdf} functionality, operations and test bench for a complete analysis of the setting of the camera parameters.}
		\item{Assembly, Integration and Testing: Build robust test system which emulated sensors, interface cards and ground software. It was used for Avionics bring ups, \textbf{On-board in Loop Simulation (OiLS)}, independent verification of telemetry, telecommand, payload interface and ground checkout.}
		\item{Mission Planning and Operations: Reviewing and making of the detailed design documents for CDR, PSR, PLR, the sequence of events, PISAT in orbit tracking and post data analysis.}
		\end{cvitems}
	}
\end{cventries}
\begin{cventries}
	\cventry
	{Open ended Project, Intel IoT Labs, P. E. S. University}
	{Smart Energy Meter using \href{https://www.arduino.cc/en/ArduinoCertified/IntelGalileoGen2}Intel Galileo Gen2}
	{Bangalore, India}
	{Jun'15 - Jan'16}
	{
		\begin{cvitems}
		\item{An Intel IoT Platform Project where Galileo Send the value to the server such that user can monitor and therefore analyze the data using MQTT protocol}
		\item{Intel ThingSpeak Cloud is Used as a server to upload the data and do further analysis via inbuilt MATLAB Tools}
		\end{cvitems}
	}
\end{cventries}
