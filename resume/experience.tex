\cvsection{Work Experience}
\begin{cventries}
	\cventry
	{\large Software Engineer 2 | Embedded \& Avionics Systems}
	{\href{https://www.boeing.co.in/}{\large Boeing India Pvt. Ltd.}}
	{\large Bengaluru, India}
	{\large Nov'19 - Present}
	{
		\begin{cvitems}
		\item{\large Designed, developed, and maintained architectures, requirements, algorithms, interfaces, and designs for \textbf{avionics} software systems across diverse product lines. Conducted feasibility studies, created robust architectures, and collaborated with cross-functional teams to ensure alignment with regulatory standards.}
		\item{\large Architected and integrated software components into fully functional systems leveraging \textbf{RTOS} or \textbf{Linux}-based platforms, utilizing a variety of programming languages and technologies. Conducted in-depth requirements analysis, architectural design, and peer reviews, ensuring full traceability and the delivery of maintainable, high-quality software solutions.}
		\item{\large Developed and maintained build infrastructure, platform services, middleware frameworks, real-time communication protocols, diagnostic tools, and mission-critical application software for \textbf{avionics} systems, ensuring high scalability, performance optimization, fault tolerance, secure data handling, and seamless integration across flight and hardware-in-the-loop simulation environments}
		\item{\large Configured custom hardware architectures and optimized Board Support Packages, including U-Boot, kernel, custom \textbf{Yocto} recipe-based root file systems, cross-compilation toolchains, and device trees, to enable efficient secure boot processes and enhance runtime performance for custom hardware platforms.}
		\item{\large Designed and implemented automated CI/CD pipelines with GitLab CI and Docker, streamlining build, integration, and deployment processes to ensure consistent and reliable software releases.}
		\end{cvitems}
	}

	\cventry
	{\large Flight Software Engineer | Integrated Avionics | Command \& Data Handling}
	{\href{http://www.teamindus.in/}{\large Team Indus (Axiom Research Labs Pvt. Ltd.)}}
	{\large Bengaluru, India}
	{\large Jan.'17 - Oct'19}
	{
		\begin{cvitems}
		        \item{\large Developing software systems for \textbf{orbital}, \textbf{descent} and \textbf{surface} phases of the soft landing lunar mission, with onboard state estimation, autonomous attitude correction, lunar terrain feature tracking, active thermal and power control, interface drivers for sensors peripherals and other interfacing cards, with limited fault detection, isolation, and recovery.}
			%\item{\large Design, develop, and/or modify engineering applications for specialized capabilities within spacecraft i.e sensor, control algorithms, fault tolerance, and mission management systems}
			%\item{Developing frameworks for regression, unit, interface and integration level of testing which involves sensor and other interface cards emulation and simulation}
			\item{\large Developed framework for \textbf{Processor in Loop Simulation (PiLS)} system emulating sensor and actuator electrical interfaces to lander avionics unit.}
%			\item{Writing, analyzing, and maintenance of software requirements for \textbf{Lander On-Board Computer(L-OBC)}\href{http://ww1.microchip.com/downloads/en/DeviceDoc/ATF697FF.pdf}, \textbf{Auxillary Flight Computer (L-AFC)} and \textbf{Rover On-Board Computer(R-OBC)}. Studying the feasibility with present architecture, providing the solution for each module development and final independent verification and validation.}
%			\item{Developing L-OBC boot architecture, custom linkers, make system and maintaining different configurations for \textbf{Atmel's RAD Hard ATF697FF} - \textbf{SPARC V8} processor operated as bare metal with \textbf{Round Robin} scheduler}
%			\item{Developing and testing embedded applications for \textbf{L-AFC} and \textbf{R-OBC} which runs on barebones Linux. Building kernel, firmware for the SoC using Petalinux and Buildroot toolchain.}
		\end{cvitems}
	}
\end{cventries}
