\cvsection{Academic Projects}
\begin{cventries}
	\cventry
	{\textbf{PISAT}\href{http://pisat.pes.edu/}- a nano-satellite project executed by P. E. S. University launched aboard \textbf{PSLV C-35} on 26th Sept'16}
	{Student Team Lead}
	{Bengaluru, India}
	{Oct.'14 - Dec.'16}
	{
		\begin{cvitems}
		\item{Involved in complete design, development, assembly, integration, and testing phase of PISAT- a nano-satellite student project funded by ISRO and PES University. Worked in following subsystems under the expertise of ex-ISRO scientists}
		\item{Develop real-time software for an imaging satellite in a component base manner which managed overall functionality such as attitude determination, control systems, telemetry, and tele-command (RTE)}
%		\item{Involved in complete design, development, assembly, integration and testing phase of \textbf{PISAT}\href{https://www.isro.gov.in/Spacecraft/pisat} - an imaging nano-satellite under ex-ISRO experts.}
%		\item{OBC and ADCS: The subsystem included building real-time software for an imaging satellite in a component base manner which managed overall functionality such as attitude determination, control systems, telemetry and telecommand (RTE) on an \textbf{Atmel AT32UC3A0512 }\href{https://www.microchip.com/wwwproducts/en/AT32UC3A0512} micro-processor with bare-metal architecture using Round Robin scheduler.}
%		\item{Payload: Develop \textbf{NanoCam C1U }\href{https://gomspace.com/UserFiles/Subsystems/datasheet/gs-ds-nanocam-c1u-17.pdf} functionality, operations and test bench for a complete analysis of the setting of the camera parameters.}
%		\item{Assembly, Integration and Testing: Build robust test system which was used for Avionics bring ups, \textbf{On-board in Loop Simulation (OiLS)}, independent verification of telemetry, telecommand, payload interface and ground checkout.}
%		\item{Mission Planning and Operations: Reviewing and making of the detailed design documents for CDR, PSR, PLR, the sequence of events, PISAT in orbit tracking and post data analysis.}
		\end{cvitems}
	}
\end{cventries}
